\documentclass[letterpaper,11pt,leqno]{article}
\usepackage{paper}
\bibliographystyle{plain}

% PDF metadata and paths
\hypersetup{pdftitle={How Informative are Job Posting Skill Measures? Evidence from Selection Decisions}}

\begin{document}

% Title and authors
\title{How Informative are Job Posting Skill Measures? \\ Evidence from Selection Decisions}
\author{%
 Nikhil George \and Ramayya Krishnan \and Rahul Telang%
 \thanks{The authors are at Carnegie Mellon University. We gratefully acknowledge support from the Block Center for Technology and Society at Carnegie Mellon University. We thank our partner firm for their collaboration and valuable data access. We also appreciate the helpful feedback received from conference and seminar participants. The analysis and conclusions presented in this paper are those of the authors.}%
}
\date{December 2024}

\begin{titlepage}
\maketitle

Information in job postings plays an increasingly central role in modern labor markets; aside from information for 
potential employees the content in them is a key part of matching algorithms, screening tools, and broader workforce 
analytics. This brings up the demand to ascertain how the measures of skills and job requirements built from job postings 
predict actual selection outcomes. We provide a first validation of the informativeness by predicting selection to internal 
jobs at a major firm from a measure of skill distance built on job postings. Skill distances built on job postings 
strongly predict selection - the probability of selection is 84\% higher when the sought job is in the lowest quintile 
of skill distance compared to a position in the highest distance quintile. And in 70\% of cases the selected candidate 
is the one with the shortest skill distance. These skill measures consistently outperform traditional employee 
characteristics in explaining selection. We build on this validation to show how internal application intensity is 
strongly correlated with average skill distance to emerging opportunities - novel insights relevant to human capital 
management and research on employee mobility in modern labor markets. Beyond validating posting informativeness through 
selection outcomes, our analysis reveals rich possibilities for research and analytics leveraging the skill content 
in job postings.

\end{titlepage}

% Main sections

\section{Introduction}


Major corporations like Google, Walmart, and IBM, along with other public and private agencies, have committed to 
skills-based hiring policies, focusing on capabilities rather than traditional credentials such as degrees or work 
experience \citep{hbr2022skillsbased, mckinsey2020future, wef2020jobs}. This shift comes at a time when employees 
increasingly pursue less fixed career paths both within and across firms, and when the skills required to perform 
modern work evolve rapidly. These forces compel firms to articulate skill requirements and job responsibilities 
with greater precision in their postings. This information is not only consumed by potential applicants—it is 
embedded in the algorithms used by platforms like Indeed and LinkedIn to curate and match opportunities to 
candidates, and ingested by sophisticated screening tools that shortlist applicant resumes based on fit against 
these requirements. Job postings and their content are also systematically harvested by specialized firms like 
Revelio and Lightcast, who package this data into analytical products that provide insights into workforce skill 
requirements across industries\footnote{The workforce analytics provided by firms like Revelio and Lightcast have 
not only supported businesses but also spurred a growing body of academic research that study trends in skill 
demands and job creation \citep{goldfarb2020artificial, azar2020concentration}, and other critical labor market 
phenomena \citep{hershbein2018recessions, forsythe2020labor, braxton2023technological}.}. These developments 
underscore the growing importance of job postings as a key source of valuable information at the analytical, 
strategic, and policy levels.

Despite this growing centrality of job posting content and significant investment in its analysis, no rigorous validation 
exists of whether posting-derived skill measures predict actual selection and mobility outcomes. While firms are 
investing in more detailed skill descriptions, and sophisticated tools are being built to extract and analyze this content, 
consumers of posting-based analytics and tools operate without clear evidence of informativeness. Legitimate concerns 
exist - postings might be aspirational rather than realistic, could be template-driven or overly generic, and might 
list ``nice to have" skills rather than true requirements. Understanding how informative these postings are for 
measuring skill requirements is crucial for the growing ecosystem of tools and analytics being built on this content. 
The ability to validate posting-derived measures against selection decisions would inform both current applications 
and future innovations.

We partner with the IT division of a major financial services firm to study the informativeness of job postings by 
linking it to actual workers and their mobility process. This setting provides an ideal laboratory - the firm's 
hiring process generates detailed position requirements through job postings, which multiple internal candidates 
consider and apply to. The firm's internal job board offers comprehensive records of both successful and unsuccessful 
applications over multiple years. Crucially, we can link the applicants to the postings of their current job and 
observe both successful and unsuccessful applications, providing the opportunity to assess the informative content 
in job postings to predict selection. By structuring a prediction problem, we can examine whether the skill 
distances derived from posting content meaningfully predict selection outcomes, thereby contributing a first of 
its kind estimate of the nature and level of informativeness in job postings; an information source of increasing 
application inside and outside the firm.

Our analysis reveals substantial predictive power in posting-derived skill measures. The probability of selection 
decreases by 84\% when comparing applications in the lowest versus highest quintile of skill distance, indicating 
posting content captures meaningful differences in worker-job fit. This predictive power is particularly evident 
when examining vacancies with multiple applicants - in 70\% of these cases, the candidate with the shortest skill 
distance was selected. The predictive relationship holds both across vacancies and within applicants (across their 
different applications). Notably, traditional employee characteristics like age, tenure, and work experience 
contribute minimally beyond the skill distance measure - suggesting the rich information in job postings dominates 
standard observable characteristics in explaining selection outcomes.

The first step in our machine learning pipeline converts the key information in a job posting to a job skill-embedding 
by mapping it in the language space of a pre-trained language model. While these embeddings capture rich semantic 
relationships in general language, our context of technical skills and responsibilities likely lies in a much 
lower dimensional space. Moreover, with relatively few selection decisions compared to the embedding dimensions, 
we need dimension reduction to enable effective learning of the distance metric. The distance metric learning, 
akin to fine-tuning the general language model's notion of similarity for our specific context, reveals meaningful 
skill relationships. This learned representation's validity is demonstrated through its strong predictive power on 
held-out selection decisions, and its application reveals novel patterns in how employees navigate opportunities 
based on skill-fit.

We apply the skill distance measure to investigate internal job application patterns, revealing another dimension of 
job posting informativeness. Those whose skills closely match available vacancies adopt a relatively passive approach 
despite higher chances of success, while employees facing larger skill distances actively pursue positions requiring 
skill development. By structuring standard posting and mobility data, we show how posting-derived analytics can generate 
actionable insights about whether reducing search frictions or investing in skill development should be prioritized. 
Such capabilities suggest a promising direction for HR analytics, a field that has seen limited advancement despite 
significant interest \citep{Tambe2019} - the combination of skill information in job postings and mobility patterns 
enables rich workforce analytics. Our analysis reveals novel patterns about internal mobility, providing evidence 
about job search behavior in an environment characterized by employed workers navigating evolving skill requirements 
- a setting distinct from traditional unemployment-focused search models. This ability to measure skill requirements 
from job postings and track mobility decisions opens new possibilities for personnel economics research to 
systematically investigate career navigation and skill development choices.

Our paper illuminates the informative content in job postings by linking their skill and responsibility descriptions 
to selection outcomes and application decisions. Through selection prediction, we provide first validation that 
posting-derived skill measures capture meaningful differences in worker-job fit - a crucial validation given the 
growing use of posting content in platforms, analytics, and research. Through our inquiry into application patterns, 
we illuminate how this posting information relates to mobility behavior. This analysis demonstrates significant 
potential for firms to gain insights about internal mobility and its relationship to skill-fit by leveraging 
posting information, while opening new research possibilities in understanding how employees navigate careers 
in environments characterized by evolving skill requirements.

To summarize, this paper illuminates how findings validate a fundamental data source driving major market decisions 
while demonstrating rich analytical possibilities. By showing posting content contains meaningful skill information 
that predicts outcomes, we inform the growing ecosystem of screening tools and analytics built on this content. The 
results challenge pessimism about HR analytics by showing how posting information can guide specific 
interventions - from reducing search frictions for well-matched workers to targeting skill development where needed. 
This posting-based approach outperforms traditional collaborative filtering methods by leveraging the rich content 
in job descriptions, opening new avenues for research in labor economics and personnel practices.

\subsection{Literature Review}

Our work advances a rich technical literature on job recommendation, worker-job matching, and skill 
representation learning---supporting platforms that facilitate job search, candidate-opening matching, and 
career trajectory modeling\footnote{The technical literature has included work on recommender systems 
\citep{shaha2012survey,siting2012job} and representation learning \citep{heap2014combining, zhu2018person, 
liu2019tripartite, bian2020learning} to support platforms facilitating job search 
\citep{heap2014combining,giabelli2021skills2job}, matching jobs to candidates \citep{zhu2018person,qin2020enhanced} 
and modeling career paths and skill recommendations \citep{maurya2017bayesian, kokkodis2021demand}.}. While these 
advances demonstrate sophisticated ways to process job posting content, their focus on platform functionality leaves 
open the fundamental question of whether derived skill measures predict actual selection outcomes. Our work provides 
the first systematic validation of these measures through observed selection decisions.

Our research contributes directly to the Information Systems literature studying algorithms that match workers to 
jobs and recommend skill development \citep{kokkodis2021demand, kokkodis2023good}, while addressing fundamental 
questions about assessing the value of data for algorithmic applications \citep{lei2024value}. Particularly relevant 
is \citet{raghavan2020mitigating}'s documentation of limited transparency in automated screening systems. By 
demonstrating that posting-derived skill measures strongly predict selection, we provide both validation of these 
measures and a framework for evaluating hiring algorithms in practice.

Our work enriches the study of internal labor markets, where different approaches to measuring employee-position 
fit have emerged. \citet{devos2024data} develops recommender systems using collaborative filtering based on observed 
transitions and employee features, while \citet{2024_Cowgill} examines team assignment mechanisms using survey-based 
skill scores validated by management. We demonstrate how the rich skill information contained in job postings can 
generate validated measures of fit that complement these approaches, requiring neither extensive mobility histories 
nor expert validation.

We provide crucial validation for a growing literature in management and economics that leverages job posting text 
to study labor market dynamics. Researchers have extracted signals from job descriptions to examine wage premia 
\citep{Bana2021}, analyze technology's impact on skill demands \citep{George2024}, and forecast effects of 
generative AI \citep{eloundou2024gpts, 2024_Acemoglu}. Our demonstration that posting-derived skill measures 
strongly predict selection outcomes provides important validation for this literature's growing use of posting 
text to measure skill requirements.

Our analysis advances research on career moves and incentives in personnel economics, where studies have documented 
patterns in internal moves \citep{bidwell2024stepping} and efficiency gains from internal hiring \citep{bidwell2011paying}. 
This literature spans from early work on career incentives \citep{baker1994internal, baker1994wage} to recent studies 
of internal labor markets \citep{tambe2020paying, huitfeldt2023internal}. By validating posting-derived skill measures 
and showing how they predict both selection and application patterns, we provide new tools for studying modern career 
navigation in environments where skill requirements evolve rapidly.

We proceed as follows: In Section \ref{sec:research_setting}, we describe the research setting, the 
firm, data and the internal mobility process. In Section \ref{sec:objective_approach}, we outline 
the conceptual ideas of the prediction problem. Section \ref{sec:machine_learning_pipeline} presents 
the machine learning pipeline, the algorithms and details of data processing and implementation. In 
Section \ref{sec:evaluation_skill_distance}, we evaluate the effectiveness of the skill distance metric 
in predicting selection outcomes in detail. In Section \ref{sec:internal_mobility_patterns} we apply 
the distance measure to new positions to study application intensity to new positions vary with skill 
fit and concluding in Section \ref{sec:conclusion_discussion}.


% Research Setting Section
\section{Research Setting}\label{sec:research_setting}

\subsection{Context}

This section outlines our research setting within the Information Technology division of a prominent U.S.-based financial services institution. The division, comprising multiple specialized units, provides technical support ranging from trading systems and wealth management platforms to enterprise-wide initiatives like system modernization and compliance reporting. Several distinctive features make this setting ideal for validating posting informativeness: precise skill specifications required for technical roles, diverse units generating varied requirements, and structured mobility processes yielding both successful and unsuccessful applications.

Within the division, specialized technical roles form the core workforce---application programmers developing trading platforms, data analysts supporting risk management, and quality assurance specialists maintaining enterprise systems. Each position demands specific technical competencies, from programming languages and analytical tools to testing frameworks and development methodologies. Such specialized requirements necessitate detailed articulation of skills in job postings, generating rich content for examining whether posting-derived measures predict selection outcomes.

A centralized job board facilitates internal mobility, where positions are posted with comprehensive skill and responsibility specifications (see Box \ref{box:vacancy_posting} for a representative example). The organization maintains minimal restrictions on internal applications, allowing employees to pursue roles aligned with their interests without prior managerial approval. Following application submission, recruiting units conduct initial screening and interviews to identify suitable internal candidates. External candidates are considered only after a specified timeframe if no appropriate internal candidate emerges.

The organization's structured internal mobility process generates substantial variation in both applications and outcomes across positions. The standardized posting and evaluation procedures, combined with comprehensive data capture of both successful and unsuccessful applications, provide the essential elements for validating posting informativeness through actual selection decisions. This combination of detailed technical postings, standardized processes, and comprehensive outcome data makes the setting particularly suitable for our research objectives.

\subsection{Data}

Our analysis draws on comprehensive data from the firm's Human Resource Information System (HRIS), which captures all job requisitions, applications, and selection outcomes. Each record includes complete job posting content, application details, and final decisions. A crucial feature for our research design lies in our ability to link internal applications to the detailed posting of the applicant's current position at application time, enabling direct comparison of skill requirements between current and sought positions.

\begin{table}[t]
    \caption{Job Postings and Internal Mobility (2018-2021)}
    \begin{tabular*}{\textwidth}{@{\extracolsep\fill}lrrr}
    \toprule
    Year & Total Positions & Internal Applications (\%) & Internal Fills (\%) \\
    \midrule
    2018 & 1,326 & 40.3 & 8.3 \\
    2019 & 1,370 & 50.1 & 17.3 \\
    2020 & 1,606 & 75.7 & 42.5 \\
    2021 & 2,240 & 62.5 & 33.9 \\
    \midrule
    Total & 6,542 & 58.6 & 27.3 \\
    \bottomrule
    \multicolumn{4}{p{\textwidth}}{\footnotesize \textit{Notes:} Data from HRIS covering 2018-2021, supplemented by 1,638 pre-2018 position postings. Internal Applications (\%) represents positions receiving at least one internal application. Internal Fills (\%) represents positions filled by internal candidates.} \\
    \end{tabular*}
    \label{tab:summary}
\end{table}

Table \ref{tab:summary} presents our dataset covering individual contributor positions posted between 2018-2021. The observation window captures 6,542 positions, with 3,836 (58.6\%) receiving internal applications and 1,789 (27.3\%) filled internally. This data is supplemented by 1,638 pre-2018 position postings, enabling broader coverage when analyzing internal applicants' current positions. The data shows a notable increase in internal mobility over the observation period, with the proportion of positions receiving internal applications rising from 40.3\% in 2018 to 62.5\% in 2021.

\begin{table}[t]
    \caption{Structure of HRIS Data}
    \begin{tabular*}{\textwidth}{@{\extracolsep\fill}lllllcc}
    \toprule
    Date & Requisition ID & Job Description & Applicant ID & Employee ID & Internal & Selection \\
    \midrule
    2021-03 & REQ2021-456 & Hadoop Developer & APP-7K89 & EMP-123 & Yes & Selected \\
    2021-03 & REQ2021-456 & Hadoop Developer & APP-7K90 & EMP-124 & Yes & Not Selected \\
    2021-03 & REQ2021-456 & Hadoop Developer & APP-7K91 & -- & No & Not Selected \\
    2021-04 & REQ2021-457 & Data Analyst & APP-7K92 & -- & No & Selected \\
    \vdots & \vdots & \vdots & \vdots & \vdots & \vdots & \vdots \\
    2021-05 & REQ2021-459 & Sr. Developer & APP-7L95 & EMP-127 & Yes & Selected \\
    2021-05 & REQ2021-459 & Sr. Developer & APP-7L96 & -- & No & Not Selected \\
    \vdots & \vdots & \vdots & \vdots & \vdots & \vdots & \vdots \\
    2021-06 & REQ2021-460 & ML Engineer & APP-7M01 & EMP-130 & Yes & Not Selected \\
    \bottomrule
    \multicolumn{7}{p{\textwidth}}{\footnotesize \textit{Notes:} Sample entries from HRIS. Employee ID is populated for internal applicants or when external candidates are selected. Job Description field contains complete posting content (see Box \ref{box:vacancy_posting}) for representative example of full posting content. Internal flag indicates current employee status.} \\
    \end{tabular*}
    \label{tab:hris_structure}
\end{table}

The structure of our HRIS data appears in Table \ref{tab:hris_structure}, which illustrates the comprehensive information available for each application. The system records application timing, position details, applicant information, and selection outcomes. Crucially, for internal applicants, employee identifiers enable linking to their current positions and corresponding job postings. This linkage capability is essential for analyzing how differences between current and sought position requirements relate to selection outcomes.

\clearpage
\begin{tcolorbox}[colback=boxbackground,colframe=boxframe,sharp corners,
title=Sample Job Description,
label=box:vacancy_posting]
\noindent \textbf{Overview}\\
We are one of the world's leading financial institutions ... and risk management products and services.

\noindent \textbf{Process Overview}\\
Build and evolve a consistent Authorized Data Source within Consumer \& Small Business Bank (CSBB) ... both the strategic and tactical analytics needs of the Consumer Bank.

\noindent \textbf{Job Description}\\
Hadoop developer for multiple initiatives. Develop Big Data Strategy and Roadmap for the Enterprise. Experience in Capacity Planning, Cluster Designing and Deployment. Benchmark systems, analyze system bottlenecks, and propose solutions to eliminate them. Develop highly scalable and extensible Big Data platform, which enables collection, storage, modeling, and analysis of massive data sets from numerous channels. Continuously evaluate new technologies, innovate and deliver solution for business-critical applications.

\noindent \textbf{Responsibilities}\\
Assists the team with the design of the architect layer to ensure re-usable metrics and attributes within the reporting layer. Responsible for creating and maintaining necessary documentation (MDR) to ensure audit readiness where necessary. Prototype improvement ideas. Work effectively with the global team. Expected to play technical leadership as an individual contributor. Articulate challenges, propose and drive solutions ...

\noindent \textbf{Mandatory Skills}\\
Extensive knowledge of Hadoop stack and storage technologies HDFS, MapReduce, Yarn, HIVE, sqoop, Impala, spark, flume, kafka and oozie. Extensive Knowledge on Bigdata Enterprise architecture (Cloudera preferred). Experience in No SQL Technologies (Cassandra, Hbase).

\noindent \textbf{Desired Skills}\\
Experience in Real time streaming (Kafka). Experience with Big Data Analytics \& Business Intelligence and Industry standard tools integrated with Hadoop ecosystem. (R, Python). Visual Analytics Tools knowledge (Tableau). Data Integration, Data Security on Hadoop ecosystem. (Kerberos). Awareness or experience with Data Lake with Cloudera ecosystem.

\end{tcolorbox}

\noindent \textit{Notes:} This is a sample vacancy posting with information about the firm, the sub-division, location etc. redacted. The posting demonstrates the rich content available for analyzing skill requirements and job responsibilities. 

Box \ref{box:vacancy_posting} demonstrates the typical depth of position postings in our dataset. As shown, each posting contains detailed specifications spanning multiple dimensions: role overview, process context, specific responsibilities, and both mandatory and desired skills. This rich content enables precise measurement of skill requirements and job characteristics, providing the foundation for validating posting informativeness.

\begin{table}[t]
    \caption{Sample Job Requisitions and Internal Applications}
    \begin{tabular*}{\textwidth}{@{\extracolsep\fill}llllc}
    \toprule
    Job Requisition No. & Current Position & Position Applied To & Date & Selected \\
    \midrule
    REQ2021-456 & Data Engineer & Hadoop Developer & 2021-03 & Yes \\
    REQ2021-456 & ML Engineer & Hadoop Developer & 2021-03 & No \\
    \vdots & \vdots & \vdots & \vdots & \vdots \\
    REQ2021-458 & Data Analyst & ML Engineer & 2021-04 & Yes \\
    REQ2021-459 & Software Engineer & Sr. Developer & 2021-05 & Yes \\
    \vdots & \vdots & \vdots & \vdots & \vdots \\
    \bottomrule
    \multicolumn{5}{p{\textwidth}}{\footnotesize \textit{Notes:} Derived from HRIS data by linking internal applicants' Employee IDs to their current positions. Both current and applied-to positions contain detailed requirements as shown in Box \ref{box:vacancy_posting}.} \\
    \end{tabular*}
    \label{tab:requisitions}
\end{table}

Table \ref{tab:requisitions} illustrates how we leverage the HRIS data structure to track internal mobility patterns. By linking internal applicants to their current positions through employee identifiers, we can observe both the positions they apply to and detailed requirements of their current roles. Among positions receiving internal applications, we identified 1,370 cases where we could observe the detailed posting of the applicant's current position at application time, forming the core sample for our analysis of posting informativeness.

The combination of detailed technical postings, comprehensive observation of application outcomes, and our ability to link current and sought positions makes this setting particularly valuable for validating posting informativeness. The rich content in job postings reflects precise skill requirements for technical positions. Moreover, access to both successful and unsuccessful applications enables robust testing of posting content's predictive power. Our unique ability to observe detailed postings for applicants' current positions at the time of application allows direct measurement of skill distances, providing compelling evidence of whether these measures meaningfully predict selection outcomes.

\section{Objective and Approach}\label{sec:objective_approach}

Our central aim is to understand the extent to which job postings convey meaningful information about the 
actual requirements of a position, specifically regarding skill fit. We investigate this by analyzing 
patterns in internal hiring decisions, under the premise that these decisions implicitly reveal the skills and 
demands prioritized by the organization.

Job postings utilize varied language to describe complex technical skills. Similar underlying requirements might 
be expressed differently across postings, while seemingly disparate descriptions could indicate similar skill needs. 
Internal mobility offers a valuable context for studying posting informativeness, as we observe both the textual 
content of position descriptions and the outcomes of employee transitions between roles. If postings effectively 
communicate skill requirements, we expect differences in their content to correlate with these transitions.

To formalize our investigation, consider an internal applicant being evaluated for a new role. Let $j_c$ denote 
the text of the applicant's current job posting and $j_v$ the text of the vacancy posting. The hiring manager 
considers various factors, some unobserved by us, represented by the feature vector $z_i$. We can model the 
selection decision as dependent on a distance measure $\delta(j_v, (j_c, z_i))$ between the vacancy and the 
applicant's profile, with selection occurring if this distance is below a threshold $\tau$:

\[
S(j_v, (j_c, z_i)) = 
\begin{cases} 
1, & \text{if } \delta(j_v, (j_c, z_i)) < \tau \\
0, & \text{if } \delta(j_v, (j_c, z_i)) \geq \tau
\end{cases}
\]


Given that we do not observe the full set of features $z_i$ considered in the actual selection process, our ability 
to perfectly predict individual hiring decisions based solely on job posting content is limited. We use the 
Area Under the Receiver Operating Characteristic Curve (AUC) to assess the extent to which the distance between 
job postings, derived from their text, aligns with observed hiring outcomes, recognizing that this metric 
provides a measure of how informative the postings are in accounting for skill fit, even with the presence 
of unobserved factors.

Our approach is guided by two key principles in representing and comparing job postings. First, we leverage 
the sophisticated language understanding capabilities of pre-trained transformer models to generate numerical 
representations (embeddings) of the posting text. The ability of these models to capture nuanced semantic 
relationships within large text corpora makes them well-suited for representing the specialized vocabulary 
of IT job descriptions, and their pre-trained nature offers significant practical advantages. Second, we 
utilize our historical selection decisions in a supervised manner to learn a distance metric between these embeddings. 
This process of metric learning allows us to fine-tune the distance measure, weighting differences in the 
embeddings according to their relevance in predicting observed hiring outcomes. This supervision allows us to 
extract signals related to the organization's implicit valuation of skills, as revealed through its hiring choices.


A critical aspect of our methodology is addressing the challenges associated with the high dimensionality of embeddings 
generated by large language models. One key motivation for dimensionality reduction is that learning a meaningful distance 
metric, which involves assigning appropriate weights to different dimensions, becomes increasingly difficult in 
high-dimensional spaces, particularly with a limited number of observed selection decisions. Furthermore, the 
embeddings of job postings, even for distinct roles within a single organization, are anticipated to exhibit a 
tendency to cluster within the broader embedding space. Therefore, we employ dimensionality reduction techniques 
to map these high-dimensional representations to a lower-dimensional space, aiming to distill the most salient 
features relevant to distinguishing between job requirements and facilitating effective metric learning.


\begin{figure}[t]
    \centering
    \begin{tikzpicture}[
        >=stealth,
        % Part I styles
        textinput/.style={cloud, draw, thick, aspect=2, minimum width=2cm, align=center},
        lmodel/.style={rectangle, draw, thick, rounded corners, fill=blue!20, minimum width=3cm, minimum height=1.5cm, align=center},
        vector/.style={rectangle, draw, thick, minimum width=2cm, minimum height=1cm, fill=green!20, align=center},
        reduction/.style={trapezium, trapezium angle=70, draw, thick, minimum width=2cm, minimum height=1cm, fill=orange!20, align=center},
        % Part II styles
        datainput/.style={rectangle, draw, thick, minimum width=6cm, minimum height=3.5cm},
        learning/.style={regular polygon, regular polygon sides=8, draw, thick, minimum size=3cm, fill=red!30, align=center},
        vectorbar/.style={rectangle, draw=none, fill=green!40, minimum width=1.5cm, minimum height=0.3cm},
        arrow/.style={thick, ->}
    ]
        % Part I Label
        \node[align=left] at (-2.5,0) {\Large Part I:};
        
        % Part I Components
        \node (jobs) [textinput] at (0,0) {Job\\Postings};
        \node (llm) [lmodel, right=2cm of jobs] {Language\\Model};
        \node (vec_rep) [vector, right=2cm of llm] {Vector\\Representations};
        \node (reduce) [reduction, right=2cm of vec_rep] {Dimension\\Reduction};
        \node (red_vec) [vector, right=2cm of reduce] {Reduced\\Vectors};
        
        % Part I Connections
        \draw[arrow] (jobs) -- (llm);
        \draw[arrow] (llm) -- (vec_rep);
        \draw[arrow] (vec_rep) -- (reduce);
        \draw[arrow] (reduce) -- (red_vec);
    
        % Part II Label
        \node[align=left] at (-2.5,-4.5) {\Large Part II:};
        
        % Part II Components
        % Data input block
        \node (data) [datainput] at (2,-4.5) {};
        
        % Column headers
        \node[text=blue] at ($(data.north west)+(1.2,-0.4)$) {Current};
        \node[text=blue] at ($(data.north west)+(3.2,-0.4)$) {Sought};
        \node[text=blue] at ($(data.north west)+(5.2,-0.4)$) {Selection};
        
        % Vertical separators
        \draw[thick] ($(data.north west)+(2.2,0)$) -- ($(data.south west)+(2.2,0)$);
        \draw[thick] ($(data.north west)+(4.2,0)$) -- ($(data.south west)+(4.2,0)$);
        
        % Sample data rows with vector bars
        \foreach \i [count=\yi] in {1,...,3}{
            \node[vectorbar] at ($(data.north west)+(1.2,-0.7-\yi*0.7)$) {};
            \node[vectorbar] at ($(data.north west)+(3.2,-0.7-\yi*0.7)$) {};
        }
        
        % Selection symbols with proper LaTeX symbols
        \node[text=green!50!black] at ($(data.north west)+(5.2,-1.4)$) {\Large $\checkmark$};
        \node[text=red] at ($(data.north west)+(5.2,-2.1)$) {\Large $\times$};
        \node[text=green!50!black] at ($(data.north west)+(5.2,-2.8)$) {\Large $\checkmark$};
        
        % Dots for continuation
        \node at ($(data.north west)+(1.2,-3.1)$) {$\vdots$};
        \node at ($(data.north west)+(3.2,-3.1)$) {$\vdots$};
        \node at ($(data.north west)+(5.2,-3.1)$) {$\vdots$};
        
        % Metric learning and output
        \node (metric) [learning, right=3cm of data] {Metric\\Learning};
        \node (dist) [vector, right=2cm of metric] {Distance\\Measure\\$d(\cdot,\cdot)$};
        
        % Part II Connections
        \draw[arrow] (data) -- (metric);
        \draw[arrow] (metric) -- (dist);
    
    \end{tikzpicture}
    \caption{Overview of our measurement approach. Part I illustrates the transformation of job posting text into vector representations: job postings are processed through a language model to create embeddings, which are then reduced to a lower-dimensional space. Part II shows how we learn a distance metric from the data: pairs of position vectors (current and sought positions) together with their selection outcomes are used to learn a distance measure that captures meaningful differences between positions.}
    \label{fig:approach_overview}
\end{figure}


The following section details our implementation pipeline. We begin by generating high-dimensional embeddings of 
job postings using pre-trained transformer models. 

\section{Machine Learning Pipeline}\label{sec:machine_learning_pipeline}

In this section, we explain the machine learning pipeline that converts vacancy posting text into continuous numerical 
vectors. We leverage the structured nature of job descriptions to extract high-signal information efficiently. Job 
descriptions, despite their length, follow a consistent format. We focus on the easily detectable Mandatory and 
Desired Skills sections, which provide concentrated information about skill and competency requirements. Using the 
Spacy natural language processing library, we parse these sections to extract key technical skills such as programming 
languages (e.g., Java, Asp.Net), web frameworks (e.g., Django), and testing tools (e.g., Selenium).

While knowledge of technologies is important, the Responsibilities section conveys crucial functional aspects of the 
role. These can vary notably even between positions requiring similar technology skills, such as developers, testers, 
and support engineers. The section typically lists tasks to be performed, which might include 'develop big data strategy,' 
'conduct code reviews,' 'engage with stakeholders' or 'application development.' We extract such task descriptions from 
job responsibilities by employing natural language processing techniques. This includes part-of-speech tagging to identify 
action verbs and dependency parsing to locate their direct objects, which are then combined to form concise verb-object 
phrases representing key tasks. By leveraging the structured format and our knowledge of the postings, we effectively 
eliminate low-signal words in the description. In the following subsections, we describe our machine learning pipeline 
which adheres to two key ideas: leveraging capabilities of a pre-trained language model and utilizing past selection 
decisions. 

\subsection{Pre-trained Embeddings}

Here we describe the approach to mapping the extracted text to the embedding space of a pre-trained language model. 
To create vector representations of the two parts we employ Sentence-BERT (SBERT), a modification of the pre-trained 
BERT network \citep{devlin2018bert}. SBERT is designed to derive semantically meaningful sentence embeddings, making 
it more efficient for comparing longer text strings than BERT embeddings, which are typically word vectors, although 
the BERT word-level embeddings vary with the context, unlike their predecessors \citep{reimers-2019-sentence-bert}. 
SBERT converts text into dense vector representations, effectively capturing the semantic meanings and relationships 
between words. This model is particularly suitable for our needs because it is pre-trained and fine-tuned for 
similarity tasks, ensuring that it can generate fixed-size embeddings that are both informative and consistent. 
The embeddings are generated using the SentenceTransformers library (model: paraphrase-MiniLM-L6-v2). By leveraging 
SBERT, we ensure that our embeddings capture the nuanced language used in job descriptions, which is critical for 
accurately representing the skills and tasks. 

For every $j_i$, we have the set of words $j_i^s$ (skills) and $j_i^t$ (tasks) and obtain SBERT embeddings 
$e_i^s$ ($384 \times 1$) and $e_i^t$ ($384 \times 1$) respectively. This separation allows for more granular 
embeddings and a focused semantic representation of each aspect. These embeddings encapsulate the semantic content 
of the job descriptions, allowing us to compare and analyze them effectively. The dual-embedding approach provides 
a comprehensive view of each position, highlighting both the required competencies and the associated responsibilities. 
This detailed representation forms the foundation for developing a robust distance metric that can accurately reflect 
the similarity between job descriptions, ultimately aiding in the assessment of internal mobility and selection probabilities.  

\subsection{Dimension Reduction}

To manage the high-dimensional embeddings generated for job descriptions, we apply Principal Component Analysis (PCA) 
to reduce the dimensionality of both \(\{e_i^s\}\) and \(\{e_i^t\}\). PCA is a technique that transforms the original 
high-dimensional data into a new coordinate system with fewer dimensions while retaining most of the original variance. 
For our embeddings, we apply PCA separately to the skill embeddings \(\{e_i^s\}\) and the task embeddings \(\{e_i^t\}\), 
retaining 90\% of the variance in each case. This process reduces the skill embeddings from 384 dimensions to 49 dimensions, 
now denoted as \(\{p_i^s\}\), and the task embeddings from 384 dimensions to 124 dimensions, now denoted as \(\{p_i^t\}\).
 Notably, the dimensionality of the skills embedding (\(\dim(p_i^s) = 49\)) is smaller than that of the tasks 
 embedding (\(\dim(p_i^t) = 124\)). This indicates that a relatively small number of components capture most of 
 the information in the skills embedding, which is expected since skills are described using a more specific vocabulary 
 compared to tasks. If we concatenate the \(\{p_i^t\}\) and \(\{p_i^s\}\) embeddings, the resulting representation is 
 still 173 dimensions, which is too high for measuring a useful distance between job postings.


To further reduce the dimensionality of our job description embeddings, we perform dimension reduction via clustering. 
The core idea is to represent each of our original embeddings in terms of its membership probabilities with respect to 
cluster centers. Specifically, we calculate the membership probability of a job description for each cluster, resulting 
in a set of \(k\) probabilities for each embedding. These \(k\) probabilities form a new vector of dimension \(k\), 
effectively reducing the original high-dimensional embeddings to a more manageable size.

We employ the Fuzzy C-Means (FCM) algorithm, which allows each embedding to belong to multiple clusters with varying 
degrees of membership. The membership probabilities are values between 0 and 1 and sum to 1, forming a probabilistic 
distribution. In our implementation, we apply FCM to both the skill embeddings \(\{p_i^s\}\) and the task embeddings 
\(\{p_i^t\}\). Let \(k_s\) be the number of skill clusters and \(k_t\) be the number of task clusters. For each job 
description, FCM generates a membership probability vector \(m_i^s\) (\(k_s \times 1\)) for skills 
and \(m_i^t\) (\(k_t \times 1\)) for tasks, indicating the degree to which each job description belongs to each cluster.

Concatenating these membership probability vectors \(m_i^t\) and \(m_i^s\) results in \(x_i\), a low-dimensional 
representation of \(j_i\). This combined vector \(x_i (k * 1)\) effectively captures the key information in \(j_i\) 
while reducing the dimensionality of the data, facilitating the development of a robust distance metric tailored to 
our specific context. This lower-dimensional representation improves the accuracy and effectiveness of our internal 
mobility predictions by preserving the most relevant information for measuring distances between job postings. The 
vectorization process described is illustratively summarized in \autoref{fig:job-vectorization}. This leaves us with 
the question of selecting the number of skill and task clusters, which we will take up after we discuss the measuring 
of distances between job postings and their informativeness regarding selection probabilities to vacancies. 


\begin{figure}[htbp]
    \centering
    \begin{tikzpicture}
        % Styles for the squares and arrows
        \tikzset{
            square/.style={
                draw,
                rectangle,
                minimum size= 2cm,
                fill=none,
                align=center
            },
            arrow/.style={
                thick,
                ->,
                >=stealth 
            }
        } 
        % Define initial position and shift for squares
        \def\initpos{0}
        \def\shift{2.75}  % Distance between centers of squares
        % Custom labels for each step in the process
        \def\labelsT{{"j_i^t", "e_i^t", "p_i^t", "m_i^t"}}
        \def\labelsS{{"j_i^s", "e_i^s", "p_i^s", "m_i^s"}}
        % Drawing the squares, split squares, and arrows
        \foreach \i in {0,...,4} {
            % Draw split square for the first four
            \ifnum\i<4
                \node[square] (square\i) at (\initpos+\shift*\i,0) {};
                \draw (square\i.south) -- (square\i.north);
                \node at ([xshift=-0.375cm] square\i.center) {$\pgfmathparse{\labelsT[\i]}\pgfmathresult$};
                \node at ([xshift=0.375cm] square\i.center) {$\pgfmathparse{\labelsS[\i]}\pgfmathresult$};
            \else
                % Draw the unsplit square for the fifth one
                \node[square] (square\i) at (\initpos+\shift*\i,0) {$x_i$};
            \fi
            % Draw arrow to next square
            \ifnum\i>0
                \draw[arrow] (square\the\numexpr\i-1\relax.east) -- (square\i.west);
            \fi
        }
    \end{tikzpicture}
    
    \vspace{1em}
    
    \begin{center}
        \small
        \begin{tabular}{lll}
            \textbf{Notation} & \textbf{Description} & \textbf{Dimensions} \\
            \hline
            $j_i^{t/s}$ & Split to task and skill components of $j_i$ & text \\
            $e_i^{t/s}$ & $\text{SBERT}(j_i^{t/s})$ & $384 \times 1$ \\
            $p_i^{t/s}$ & $\text{PCA}(e_i^{t/s})$  & $124 \times 1$ (tasks), $49 \times 1$ (skills) \\
            $m_i^{t/s}$ & $\text{FuzzyC}(p_i^{t/s})$  & $k_t \times 1$ , $k_s \times 1$  \\
            $x_i$ & $m_i^t \oplus m_i^s$ & $k_t + k_s \times 1$ \\
        \end{tabular}
    \end{center}
    
    \caption{This is a pictorial representation of the vectorization process that captures the information in a job posting to a numerical vector. The table provides a summary of the notation and dimensions used at each step of the process.}
    \label{fig:job-vectorization}
\end{figure}



\subsection{Distance Metric Learning} 

Now that we have represented the textual job descriptions in a vector format, we test whether these representations capture 
meaningful skill information by learning a distance metric that predicts selection outcomes. In formulating this distance, 
our two major ideas included leveraging the capabilities of pre-trained language models and using past selection decisions 
in a supervisory capacity to learn which aspects of posting content are most informative for predicting selection.

The ability to learn a custom metric is advantageous because it allows us to create a distance function tailored to the 
specific characteristics and selection patterns of our organization. Traditional distance metrics, such as Euclidean 
distance, treat all dimensions equally and do not take into account the nuances of job descriptions and the importance 
of different skills and responsibilities in the selection process. By learning a custom distance metric, we can weigh 
the various components of \(x_i\) according to their relevance in the actual hiring decisions. For instance, if a vacancy 
requires expertise in Mainframe development but also involves some tasks related to preparing reports and using 
presentation software, the lack of expertise in Mainframe development would significantly reduce the chances of 
being selected, whereas the latter skills, which are easier to acquire, would have less impact. This approach leads 
to a more accurate and informative measure of job similarity. In other words, our goal is to craft a distance that is 
specifically adapted to the problem of assessing skill adequacy for filling the vacancy.

We will focus on learning a distance from the family of Mahalanobis distances, which are parameterized by a 
matrix \(M\). The core idea is that dimensions contributing less to the distance should receive lower weight. 
For example, if a vacancy requires expertise in Mainframe development but only some competence in word processing, 
the latter skill will be weighted less in the distance measure. If the matrix is diagonal, the distance calculation 
represents a stretch for each dimension. Irrelevant covariates will be compressed so that their values are always 
effectively zero. Highly relevant covariates will be stretched so that for two units to be considered a match, 
they must have very similar values for those covariates. In this way, diagonal matrices lead to very interpretable 
distance metrics. If the Mahalanobis distance matrix is not constrained to be diagonal, then it induces a stretch 
and rotation, leading to more flexible albeit less interpretable notions of distance. The distance function we aim 
to learn is defined as:

\begin{equation}
d(x_v, x_c; M) = \sqrt{(x_v - x_c)^T M (x_v - x_c)}
\end{equation}
Here, \(x_v\) and \(x_c\) are the vector representations of the competencies required at the sought vacancy \(v\) and 
the current job \(c\), respectively and $M$ is a positive semi-definite matrix.

Our dataset \(A\) consists of tuples \((x_v, x_c, S_{v,c})\), where \(j_v\) is the current job description, \(j_c\) is 
the sought vacancy description, and \(S_{v,c}\) is the binary selection indicator. Our focus is on a distance that can 
discriminate between skill distances in the set of \textit{plausible} applications, whose empirical counterpart is \(A\). 
In other words, we want \(M\) such that the distance between cases that resulted in rejection is larger compared to cases 
that resulted in selection. We utilize the past selection and rejection data to inform our metric learning process. 
Specifically, we adopt the Mahalanobis Metric for Clustering (MMC) algorithm by \citep{Xing2002} to learn a distance 
function that can effectively differentiate between selected and non-selected candidates. The optimization problem solved 
is the following: 

\begin{align*}
\text{Maximize:} \quad & \sum_{(v,c): S_{v,c} = 0} d(x_v, x_c; M) \\[1em]
\text{Subject to:} \quad & \sum_{(v,c): S_{v,c} = 1} d(x_v, x_c; M)^2 \leq 1 \\
& M \succeq 0
\end{align*}


Adapting to our context, we are maximizing the sum of distances for cases that did not result in selection. The constraints 
are regularity constraints that ensure the data is not collapsed to a point and the distance satisfies the triangle 
inequality. This approach allows us to craft a distance metric \(d(x_v, x_c)\) that effectively captures the skill and 
competency fit between positions, utilizing the historical selection data to guide the learning process. Our dataset \(S\) 
comprises 1060 observations, which forms 75\% of our total data points on selection decisions. The implementation of this 
approach uses Python 3.5, with the MMC algorithm available in the sci-kit learn library, requiring no special computing 
resources. In the following subsection, we will discuss the parameter tuning process for determining the optimal length of 
embeddings, which in turn will define the size of the matrix \(M\).




\subsection{Parameter Tuning} 

To implement the algorithm, we need to determine the optimal number of clusters ($k_t$, $k_s$) for the reduced task and 
skill embeddings, which will define the final representation $x_i$ of a job description. This selection must be data-driven 
to ensure the most accurate distance metric for predicting selection probabilities. Using our dataset of 1,370 internal a
pplications and their outcomes (excluding rejections in favor of other internal applicants), we employ 5-fold cross-validation 
on the application dataset $A$. We iteratively test combinations of $k_s$ and $k_t$, where $k_s, k_t \in \{2, 3, ..., 20\}$, 
subject to the constraint $k_s + k_t < 25$. This constraint is a practical one: with limited training data, we cannot allow 
the dimensionality of M to grow too large, as this could lead to overfitting. For each combination, we compute the average 
Area Under the Receiver Operating Characteristic Curve (AUC).

The results, reported in Figure \ref{fig:AUC}, show that the best model achieves an AUC of 0.62 with $k = 20$, where 
$k_s = 17$ (skill clusters) and $k_t = 3$ (task clusters). This cross-validation process ensures that the chosen number 
of clusters provides a representation that effectively captures the underlying patterns in the data, leading to a more 
accurate and informative distance metric for predicting internal mobility and selection outcomes.  

\begin{figure}[htb]
    \centering
    \includegraphics[width=0.75\textwidth]{new_img/chart.png}
    \caption{5 fold AUCs computed at different dimensionality of $k$}
    \label{fig:AUC}
\end{figure}


The AUC of 0.62 indicates that skill distances derived purely from job posting text have meaningful predictive power - 
the probability that a randomly chosen selected application will be ranked as better fit than a randomly chosen rejected 
application is substantially higher than chance. This demonstrates that job postings contain real information about position 
requirements that predicts selection outcomes. As we show in subsequent sections, this validation of posting informativeness 
opens new possibilities for understanding skill requirements and worker-job fit.

To recap, we developed a machine learning pipeline to test whether job posting content meaningfully captures differences in 
position requirements by predicting selection outcomes. Our approach leverages the capabilities of pre-trained language models
and the opportunity to take a supervised approach by utilizing past selection decisions. This combination of rich text 
representations and actual selection outcomes demonstrates that postings contain meaningful information about skills and 
requirements, providing a foundation for deeper analysis of how workers and firms navigate evolving skill demands.



\section{Evaluation: Does Skill Distance Predict Selection?}\label{sec:evaluation_skill_distance}

This section evaluates whether our job postings based skill-distance measure meaningfully captures skill requirements by testing how well our derived distance metric predicts 
selection of internal candidates for new positions. Our test set comprises 308 cases where internal applicants applied to vacancies (approximately 25\% of our total cases). 
We examine whether skill distances computed purely from posting content capture predictable variation in selection outcomes. It's important to note that the information 
contained in an applicant's job description ($j_c$) may not fully capture the skills and abilities of that position, let alone those of the candidate. This potential measurement 
loss informs our approach. Our primary interest lies in understanding the extent to which posting content predicts selection, and in assessing the relative contribution of the 
skill distance measure compared to a few other observed features.

\subsection{Skill Distance and Selection Probability}

To assess the relationship between skill distance and selection probability, we employ a logistic regression to quantify the association:

\begin{equation}
\text{logit}(P(\text{S}{c,v} = 1)) = \beta_0 + \beta_1 \times Q_k(d{j_c, j_v})
\end{equation} 

Here, \( Q_k(d_{j_c, j_v}) \) represents the quintile of the skill distance between the applicant's current job (\( j_c \)) and the vacancy (\( j_v \)). We discretize the 
distance into a categorical variable, as the relationship between selection probability and distance is unlikely to be simply linear. This discretization offers stability 
and robustness to extreme values while retaining easy interpretation. \autoref{tab:logistic_regression} shows that the coefficient for skill distance is negative and 
significant \((\beta_1 = -0.3268, \, p < 0.01)\), indicating that applicants with lower skill distance to a vacancy have a higher likelihood of being selected.


\begin{table}[h]
    \centering
    \caption{Logistic Regression Results for Skill Distance and Selection Probability}
    \renewcommand{\arraystretch}{1.2} 
    \begin{tabular}{lcc}
    \hline
    \textbf{Variable} & \textbf{Coefficient} & \textbf{Std. Error} \\
    \hline
    Intercept & 1.1171*** & 0.265 \\
    Quintile & -0.3268*** & 0.086 \\
    \hline
    Observations & \multicolumn{2}{c}{308} \\
    Pseudo R-squared & \multicolumn{2}{c}{0.03551} \\
    \hline
    \multicolumn{3}{l}{\footnotesize{*** p$<$0.01}} \\
    \end{tabular}
    \label{tab:logistic_regression}
\end{table}

The regression suggests that, on average, an applicant-vacancy pair in the lowest skill distance quintile has an 84\% higher probability of selection compared to a pair in 
the highest quintile. \autoref{fig:skill_distance_probability} illustrates the association between skill distance quintile and predicted selection probability.


\begin{figure}[h]
    \centering
    \includegraphics[width=0.75\textwidth]{new_img/pp.png}
    \caption{Selection Probability by Skill Distance Quintile}
    \label{fig:skill_distance_probability}
\end{figure}

While skill distance accounts for a small component of the variation in selection decisions, its ability to capture differences in selection probability is unambiguous. 
We will now exploit the structure in our test sample to examine the association. 


\subsection{Skill Distance: Within-Vacancy and Within-Applicant Variation}

When multiple internal candidates apply to the same vacancy, we have an opportunity to verify if selection among different applicants for the same position is predicted 
by their respective skill distances. Correlation between skill distance and selection measured at each vacancy level would convey a more direct link of the measured distance. 
Similarly, we have cases where an applicant has applied to different vacancies. This allows us to focus on applicant-level selection data to examine whether the applicant 
has a higher selection probability for vacancies to which they have a low skill distance. Focusing on a single applicant allows us to control for unobserved factors that 
influence their selection chances across vacancies, such as broadly relevant skills they have acquired or performance in their current job. While we have already established 
a notable negative association between skill distance and selection, these analyses would further demonstrate that the measured skill distance captures significant selection 
variation at both the applicant and vacancy levels, thereby reinforcing the measure's informativeness.



\subsubsection{Within-Vacancy Selection}

To assess the predictive power of the skill distance metric, we evaluated the proportion of vacancies where the applicant with the lowest skill distance was selected. Our 
analysis encompassed 261 observations across 110 vacancy groups from our test set, each group comprising a vacancy with multiple internal applicants and one selection. The 
analysis revealed that the candidate with the lowest skill distance was chosen in 69.92\% of cases, suggesting the metric's reliability in predicting selection outcomes. We 
further analyzed the role of skill distance in the selection probability among applicants competing for the same vacancy using a conditional logit model. The model incorporated 
vacancy-specific fixed effects ($\alpha_v$) to concentrate on within-vacancy variation. The results indicated that being in a higher skill distance quintile decreased the 
probability of selection by 35\% ($\beta = -0.4293$, $p < 0.01$). The initial exploratory analysis demonstrates the skill distance metric's capacity to predict selection 
outcomes in a majority of cases. Building on this, the more formal conditional logit model enables a rigorous examination of the role of skill distance in differentiating 
applicants within the same vacancy. Together, these findings highlight the relevance and importance of the skill distance measure in the selection process. 

\begin{equation}
\text{logit}(P(\text{S}_{v,c} = 1)) = \alpha_v + \beta Q_k(d_{j_c,j_v})
\end{equation}

\begin{table}[h]
    \centering
    \caption{Conditional Logit Model Results for Within-Vacancy Analysis}
    \renewcommand{\arraystretch}{1.2}
    \begin{tabular}{lcc}
    \hline
    \textbf{Variable} & \textbf{Coefficient} & \textbf{Std. Error} \\
    \hline
    Quintile & -0.4293*** & 0.122 \\
    \hline
    Observations & \multicolumn{2}{c}{261} \\
    Groups & \multicolumn{2}{c}{110} \\
    \hline
    \multicolumn{3}{l}{\footnotesize{*** p$<$0.01}} \\
    \end{tabular}
    \label{tab:within_vacancy}
\end{table}

This within-vacancy analysis strengthens our confidence that skill distance is a crucial factor in differentiating among applicants, rather than merely reflecting individual 
selection probabilities. These results underscore the relevance and importance of the skill distance measure in the selection process. 



\subsubsection{Within Applicant Selection}

Now, we analyze the ability of our skill distance measure to account for selection decisions within an applicant's multiple vacancies. We again employ a conditional logit model, 
this time incorporating applicant-specific fixed effects ($\alpha_c$).


\begin{equation}
\text{logit}(P(\text{S}_{c,v} = 1)) = \alpha_c + \beta Q_k(d_{j_c,j_v})
\end{equation}

The test dataset now comprises 111 observations from 33 applicant groups, each applying to multiple vacancies. The results reveal that applicants in higher skill distance 
quintiles experience a 50\% reduction in selection probability (\(\beta = -0.695\), \(p = 0.017\)). This demonstrates that the measured skill distance effectively accounts 
for selection decisions within applicants. 


\begin{table}[h]
    \centering
    \caption{Conditional Logit Model Results for Across-Applicant Analysis}
    \renewcommand{\arraystretch}{1.2}
    \begin{tabular}{lcc}
    \hline
    \textbf{Variable} & \textbf{Coefficient} & \textbf{Std. Error} \\
    \hline
    Quintile & -0.695** & 0.290 \\
    \hline
    Observations & \multicolumn{2}{c}{111} \\
    Groups & \multicolumn{2}{c}{33} \\
    \hline
    \multicolumn{3}{l}{\footnotesize{** p$=$0.017}} \\
    \end{tabular}
    \label{tab:across_applicant}
\end{table}


These patterns demonstrate that the skill distance from an applicant's current job provides valuable information for predicting selection to new internal vacancies. While 
selection typically depends on an applicant's skill-set and preparedness for the new position's demands, our findings reveal that a distance measure constructed from job 
postings and past selection decisions correlates strongly with observed selection patterns. This correlation opens an analytical window into the firm, linking skills to 
internal mobility. Before exploring these links we examine how important is the skill distance measure relative to other observed features in predicting selection to vacancies.



\subsection{Other Observed Features}

While our paper primarily focuses on leveraging information from job postings and selection decisions, we recognize the importance of evaluating other predictors of 
internal candidate selection. To enhance our understanding of selection patterns, we examine additional features available in the firm's information system. We compare 
the performance of our skill distance metric against other observed applicant characteristics, including tenure at the organization, total work experience, job category, 
vacancy (job sought) category, and proficiency rating. Table \ref{tab:other_features} presents summary statistics for these features. 

\begin{table}[h]
    \centering
    \caption{Summary Statistics -- Other Observed Features}
    \begin{tabular}{l c c c c}
        \toprule
        \textbf{Notation} & \textbf{Description} & \textbf{Unit} & \textbf{25th Percentile} & \textbf{75th Percentile} \\
        \midrule
        $t_c$ & Tenure at $c$ & days & 661 & 1346 \\
        $T_c$ & Total experience & days & 1927 & 3009 \\
        $l_c$ & Job category & 4 levels & 2 & 3 \\
        $l_v$ & Vacancy category & 4 levels & 2 & 3 \\
        $p_c$ & Proficiency rating & 4 levels & 2 & 2 \\
        \bottomrule
    \end{tabular}
    \label{tab:other_features}
\end{table}

The data shows notable differences among applicants. Tenure ranges from about 2 years to 3.5 years, while total experience spans from roughly 5 years to 8 years at the 25th and 75th percentiles. Job and vacancy categories both range from level 2 to 3, indicating movement across organizational tiers. Proficiency ratings, however, remain consistent at level 2 for at the 25th and 75th percentiles. These observed features, as shown in Table \ref{tab:other_features}, display clear variation across the applicant pool. Their potential to explain selection patterns merits further exploration, complementing our primary skill distance metric.


To assess the relative importance of the skill distance measure  we employ Random Forest classifiers. The well known ensemble learning algorithm captures complex interactions between variables. We train two Random Forest models on our training dataset: one incorporates all features including skill distance, while the other excludes skill distance. This method isolates and quantifies the predictive power of the skill distance metric relative to other observed features. We evaluate these models on our test set and will again employ the AUC metric. Table \ref{tab:model_comparison} presents the AUC scores for both models on the test set.

\begin{table}[h]
    \centering
    \caption{Model Comparison: Random Forest Classifiers with and without Skill Distance}
    \renewcommand{\arraystretch}{1.2} 
    \begin{tabular}{lcc}
    \hline
    \textbf{Model} & \textbf{AUC Score} \\
    \hline
    Full Model (with Skill Distance) & 0.600 \\
    Model without Skill Distance & 0.500 \\
    \hline
    \end{tabular}
    \label{tab:model_comparison}
\end{table}


The full model, which includes skill distance, achieves an AUC of 0.600, while the model without skill distance yields an AUC of 0.500 - equivalent to a classifier with no predictive ability. This stark contrast in performance underscores the remarkable predictive power of the skill distance measure in internal selection processes. Despite the clear variation in other observed features such as tenure, total experience, job categories, and proficiency ratings, skill distance alone accounts for nearly all predictable variation in selection outcomes. Among this set of observed features, the skill distance measure between an applicant's current job and the sought vacancy contributes virtually all the predictive power for selection decisions in our setting.

This finding is particularly noteworthy given that we derived our measure from job descriptions, which are generally less sensitive than other HR data. While firms often possess additional potentially predictive features, these may be subject to privacy concerns or legal restrictions. Our approach demonstrates that valuable insights can be derived from less sensitive data sources, advancing the field of people analytics while maintaining ethical data practices. Moreover, the granularity of our skill distance measure appears to capture nuanced differences between roles that categorical variables like performance ratings may miss, especially when applicants cluster in just one or two groups.


The significant predictive power of our analysis validates that job postings contain meaningful information about position requirements and worker-job fit. While firms often possess additional potentially predictive features, our results demonstrate that posting content alone captures substantial signal about skill requirements. That this prediction comes from job descriptions, which are generally less sensitive than other HR data, suggests postings are an underutilized source of information about skill requirements in labor markets. This validation of posting informativeness through selection outcomes supports their growing use in labor market analytics and tools while highlighting opportunities for deeper analysis of skill requirements and worker-job matching. 


\section{Decoding Internal Mobility Patterns}\label{sec:internal_mobility_patterns}


Having developed a measure of skill distance between positions based on job postings and validating it, we explore how 
this can be applied to understand mobility patterns within the firm. The measure quantifies the distance between skills 
and tasks enumerated in job postings and assesses selection chances of candidates for new vacancies. In this section, 
we apply the skill distance metric to study how internal candidates' application behaviors vary based on their 
prospects at the time of application. By capturing predictable variation in selection, this measure enables a more 
nuanced examination of internal transitions and application patterns. The firm can now pose more sophisticated queries 
about internal mobility. For instance, it can now assess for any employee the proportion of new vacancies where their 
skill distance suggests a high probability of selection based on historical patterns. Central to our empirical 
investigation is the average skill distance measure, which assesses an employee's prospects for movement within the firm. 
From an internal applicant's perspective, if many vacancies share skills similar to their current job, their prospects are 
good; if most vacancies are distant, their prospects are weaker. We first develop this average skill similarity measure, 
then proceed to a detailed empirical analysis of application patterns and internal mobility. This analysis opens up new 
possibilities for understanding and supporting workforce development.




\subsection{Average Skill Distance}

To formalize the concept of average skill distance introduced above, we define the following metric:

\begin{equation}
    \overline{d}_{j_c, t} = \frac{1}{|J_t|} \sum_{j_v \in J_t} d(j_c, j_v)
\end{equation}

where $\overline{d}_{j_c, t}$ represents the average distance to job postings at time $t$, $J_t$ is the set of job postings in time window $t$, and $d(j_c, j_v)$ is the distance between the current job $j_c$ and a vacancy $j_v$. In our empirical analysis, we compute this measure for each internal applicant, considering all vacancies in a 30-day window prior to their observed application. This approach allows us to quantify an employee's prospects at the time of application, as discussed in the preceding paragraph. The distribution of $\overline{d}_{j_c, t}$, illustrated in \autoref{fig:s_dist_hist}, reveals notable heterogeneity in applicants' average skill distances, providing a foundation for our subsequent analysis of application patterns and internal mobility.

\begin{figure}[h]
    \begin{center}
        \begin{minipage}{\textwidth}
            \centering
            \includegraphics[width=0.8\textwidth,height=0.4\textheight]{new_img/histogram_prosp.png}
            \caption{Distribution of Average Skill Distance (\(\overline{d}_{j_c, t}\))}
            \label{fig:s_dist_hist}
        \end{minipage}
    \end{center}
\end{figure}




\subsection{Empirical Analysis}

Our focus is to develop an understanding of how skill match to vacancies, which we can now measure, affects application 
patterns. Applying is a key step in the internal mobility process and the stage of the search process that can be observed. 
We will look at two aspects of the application process: its directness and intensity. Directness corresponds to whether 
the applicant is applying to positions where they have a higher probability of securing a job, measured using the skill 
distance. Intensity captures how many positions the applicant applies to. Both directedness and intensity improve the 
probability of moving to a new position within the firm. From the firm's perspective, understanding how application 
behavior predictably varies enables them to tweak processes to support employee mobility within the organization. Our 
empirical investigation focuses on three key patterns in application behavior as they relate to the average skill distance 
measure, and we present our findings for each of these patterns in the following sub-sections.


\subsubsection{Average Skill Distance and Application Intensity}

The relationship between an applicant's average skill distance to available vacancies and the positions they choose to 
apply for provides crucial insights into internal mobility patterns. We investigate whether a higher average skill 
distance is associated with applying to closer or farther positions in terms of skill requirements. This analysis 
helps us understand how employees navigate their career paths within the firm when faced with varying degrees of skill 
alignment with available opportunities.

Applicants with a high average skill distance to vacancies face a potentially smaller set of opportunities that closely 
align with their existing skillset. This situation presents a dilemma: do these employees constrain their applications 
to fewer, more closely aligned positions, or do they make more exploratory applications to positions that require 
significant skill development? Conversely, applicants whose skillsets are closer to the available vacancies might 
be less inclined to seek positions that require learning new technologies or skills.

To empirically examine how skill remoteness affects application behavior, we estimate the following model:

\begin{equation}
    d_{j_v, j_c} = \beta_0 + \beta_1 \overline{d}_{j_c, t} + \epsilon
\end{equation}

$d_{j_v, j_c}$ is the standardized skill distance between the current job $j_c$ and the job applied for $j_v$, 
and $\overline{d}_{j_c, t}$ is the standardized average skill distance to vacancies.


\begin{table}[h]
\centering
\caption{Average Skill Distance and Application Intensity} 
\renewcommand{\arraystretch}{1.2}
\begin{tabular}{lcc}
\hline
\textbf{Variable} & \textbf{Coefficient} & \textbf{Std. Error} \\
\hline
Constant & 2.945e-16 & 0.022 \\
$\overline{d}_{j_c, t}$ (std) & 0.6093*** & 0.025 \\
\hline
R-squared & \multicolumn{2}{c}{0.371} \\
\hline
\multicolumn{3}{l}{*** p$<$0.01, ** p$<$0.05, * p$<$0.1} \\
\end{tabular}
\label{tab:skill_remote_app}
\end{table}

The results presented in Table \ref{tab:skill_remote_app} reveal a strong positive association 
between $\overline{d}_{j_c, t}$ and $d_{j_v, j_c}$. Specifically, we find that a one standard deviation 
increase in average skill distance to vacancies is associated with a 0.6093 standard deviation increase 
in the distance to the position applied for. This substantial and statistically significant effect suggests 
that applicants with higher skill remoteness are more likely to apply to positions that are farther away in 
terms of skillset. The application patterns here confirm that when faced with a higher average distance to 
available vacancies, applicants make more exploratory choices. Rather than constraining their applications 
to a narrower set of closely aligned positions, employees appear to broaden their search, potentially seeking 
opportunities that require attaining new skills and capitalize on the firm's knowledge of the applicant's 
performance in the current role which cannot be easily conveyed when applying to a position outside the firm. 
We already know that applying to a more distant position reduce selection chances, but will examine this more closely. 




\subsubsection{Average Skill Distance and Selection Probability}

We've already seen that skill distance is inversely linked to selection probability, and that applicants facing 
higher average skill distance to the set of vacancies tend to apply to more distant positions. Connecting these 
dots, we expect that applicants dealing with higher average skill distances to vacancies are less likely to get 
selected. We examine this here:


\begin{equation}
\text{logit}(P(S_{v,c} = 1)) = \beta_0 + \beta_1 \times \mathbb{I}[\text{above}]_{j_v,j_c}
\end{equation} 

$S_{v,c} = 1$ is a binary variable indicating whether an applicant with current job $j_c$ is selected for the 
job vacancy they applied to $j_v$. The variable $\mathbb{I}[\text{above}]_{j_v,j_c}$ is an indicator function 
that equals 1 if the skill distance between $j_c$ and $j_s$ is above the median, and 0 otherwise. 
Table \ref{tab:selection_prob} presents the results of this logistic regression:

\begin{table}[h]
\centering
\caption{Logistic Regression Results: Impact of Skill Distance on Selection Probability} 
\renewcommand{\arraystretch}{1.2} % Increased spacing
\begin{tabular}{lcc}
\hline
\textbf{Variable} & \textbf{Coefficient} & \textbf{Std. Error} \\
\hline
Constant & -0.0061 & 0.078 \\
Above Median & -0.4083*** & 0.112 \\
\hline
Log-Likelihood & \multicolumn{2}{c}{-895.62} \\
\hline
\multicolumn{3}{l}{\footnotesize{*** p$<$0.01, ** p$<$0.05, * p$<$0.1}} \\
\end{tabular}
\label{tab:selection_prob}
\end{table}

The results align with our expectations. Applicants with above-median skill distance to the vacancies have 
a 33.5\% lower likelihood of being selected into their position. With this we clearly see that the avg. skill distance 
to the vacancies affect the applicant's ability to secure an internal position with the same effort. The natural 
follow up and closely aligned question is of the intensity of the application process which we take up next.


\subsubsection{Average Skill Distance and Application Intensity}

We will now examine whether the application intensity—the number of vacancies an applicant applies to—varies with 
their average skill distance to the vacancy set. We already know that applicants faced with a high $d_{c,t}$ apply 
to more distant positions and are less likely to be selected. Our goal here is to inquire whether they also apply 
to more positions. For this analysis, we slightly modify our approach to constructing application windows. We define 
each window as a sequence of applications where the gap between any two consecutive applications does not exceed 
30 days. This allows for windows that may span longer than 30 days, provided no two consecutive applications 
within the window are more than 30 days apart. We estimate the following regression:

\begin{equation}
    A_{i,t} = \beta_0 + \beta_1 \mathbb{I}[\text{above}]_{i,t} + \epsilon
\end{equation}

Here, $A_{i,t}$ is the number of applications person $i$ submits in time window $t$, 
and $\mathbb{I}[\text{above}]_{i,t}$ indicates if their skill remoteness is above the median.


\begin{table}[h]
\centering
\caption{Average Skill Distance and Application Intensity}
\renewcommand{\arraystretch}{1.2} % Increased spacing
\begin{tabular}{lcc}
\hline
\textbf{Variable} & \textbf{Coefficient} & \textbf{Std. Error} \\
\hline
Constant & 1.8210*** & 0.137 \\
Above Median & 0.5334*** & 0.205 \\
\hline
Observations & \multicolumn{2}{c}{637} \\
R-squared & \multicolumn{2}{c}{0.011} \\
\hline
\multicolumn{3}{l}{\small{*** p$<$0.01, ** p$<$0.05, * p$<$0.1}} \\
\end{tabular}
\label{tab:skill_remote_intensity} 
\end{table}


Table \ref{tab:skill_remote_intensity} shows that applicants with above-median skill remoteness apply to about 1.5 
times more positions on average. Faced with diminished prospects for their skillset within firm opportunities, 
they adopt a more exploratory approach when seeking a new position. Putting all our findings together, we get a 
clearer picture of how skill remoteness to the new vacancies shape internal job applications within the firm. 
Applicants with higher skill distance to the vacancies tend to apply to jobs that are more distant in terms of skills, 
face lower chances of being selected, and submit more applications on average. In other words our skill distance measure 
has allowed to bring out distinct difference in the application patterns to new vacancies. With this understanding, 
the natural follow up question is how we can fine-tune internal mobility processes in the firm? which we take up next.




\subsection{From Mobility Patterns to Practice}

In a technology-focused firm, understanding and supporting internal mobility requires a clear view of how employees' 
current skills align with emerging requirements. Our analysis of job posting content reveals striking patterns in 
how employees navigate opportunities based on their skill-fit - patterns that would be difficult to discern without 
validated measures of skill distance. Employees whose skills closely match emerging opportunities often adopt a 
passive approach despite higher chances of success, while those facing larger skill distances pursue positions 
more actively but with lower success rates.

These empirically-documented patterns point to a clear quantitative framework for workforce development. When employees' 
skills align well with vacancies but application rates are low, the primary barrier appears to be search friction rather 
than skill gaps. As \autocite{invisiblehand} note, managerial initiative in surfacing opportunities can be crucial in 
large firms. Conversely, when employees face high average skill distances to vacancies and apply actively but unsuccessfully, 
the evidence suggests skill development needs rather than search frictions limit mobility. This ability to distinguish 
between friction and skill gap challenges using posting content enables firms to develop targeted interventions.

While our analysis illuminates these distinct patterns, moving from measurement to practice requires careful consideration. 
For employees with low average skill distance who do apply, the data supports proactive opportunity identification. 
However, for those not observed applying, questions remain about how best to surface relevant opportunities. Similarly, 
the application patterns of employees facing larger skill distances provide valuable signals about career directions 
they see as feasible or desirable - opening new possibilities for studying how employees navigate skill requirements 
in modern careers. This suggests rich possibilities for building systematic, evidence-based approaches to workforce 
development by leveraging the skill information in job postings. Rather than relying on generic mobility programs, 
firms can use posting content to quantify where reducing search frictions versus supporting skill development would 
be most valuable, while gaining deeper insight into how employees approach career development in environments of 
evolving skill demands.



\section{Conclusion and Discussion}\label{sec:conclusion_discussion}

Our major contribution lies in providing a rigorous empirical test and validation of the informativeness of job posting content, 
particularly for technical roles in the modern digital economy. Analyzing internal mobility within a major corporation's technology division, 
we demonstrate that skill descriptions in job postings possess significant predictive power for selection decisions. Crucially, our methodology 
leverages information from both the applicant's current role and the sought-after position, revealing that the calculated skill distance strongly 
discriminates between candidates' fit to new opportunities. Indeed, the probability of selection is 84\% higher when comparing applications with 
the closest versus furthest skill distance. This finding, derived from observing both successful and unsuccessful applications, provides robust 
validation of job postings as a meaningful signal of actual skill requirements, at least within the context of technology-related roles in large organizations.

Furthermore, our work illuminates how this validated informativeness of job postings unlocks key insights into internal mobility and 
human capital analytics within the firm. By establishing that posting content reliably reflects skill demands, we demonstrate its value 
as a primary data source for understanding employee transitions and career pathways. The measured skill distances between current and 
potential positions serve as a powerful lens for identifying areas where search frictions impede well-matched candidates and where genuine 
skill gaps exist relative to emerging opportunities. The observed heterogeneity in application intensity, directly linked to skill alignment 
with current vacancies, further underscores how employees actively navigate the internal market, balancing immediate fit with aspirations for 
skill development.

The demonstrated informativeness of job requirements validates and strengthens the broader trend in systematic collection and analysis of job postings data, 
both within and beyond organizational boundaries. This trend, fueled by specialized firms and academic researchers alike, seeks to understand the dynamics of 
the labor market and provide strategic intelligence to firms. Our findings underscore that firms' internal repositories of job postings, alongside increasingly 
available external data on skill requirements across their competitive landscape, represent a valuable, and now empirically validated, resource. This access 
enables a more nuanced understanding of evolving skill demands, informing strategic decisions related to talent acquisition, training, and workforce planning.

Our results suggest firms can move beyond conventional HR metrics to develop validated measures of worker-job fit, leveraging readily available posting data 
to guide specific interventions and foster more effective internal mobility. This capability to link rich posting content to tangible selection outcomes 
firmly positions job postings as a powerful, and often underutilized, source of intelligence for navigating the complexities of human capital management 
in modern, skill-driven labor markets.

\bibliography{references}

\appendix
\section{Sample Job Description}
\section{Sample Job Description}

\begin{tcolorbox}[colback=boxbackground,colframe=boxframe,sharp corners]

\noindent \textbf{Overview}\\
We are one of the world’s leading financial institutions ... and risk management products and services.

\noindent \textbf{Process Overview}\\
Build and evolve a consistent Authorized Data Source within Consumer \& Small Business Bank (CSBB) ... both the strategic and tactical analytics needs of the Consumer Bank.

\noindent \textbf{Job Description}\\
Hadoop developer for multiple initiatives. Develop Big Data Strategy and Roadmap for the Enterprise. Experience in Capacity Planning, Cluster Designing and Deployment. Benchmark systems, analyze system bottlenecks, and propose solutions to eliminate them. Develop highly scalable and extensible Big Data platform, which enables collection, storage, modeling, and analysis of massive data sets from numerous channels. Continuously evaluate new technologies, innovate and deliver solution for business-critical applications.

\noindent \textbf{Responsibilities}\\
Assists the team with the design of the architect layer to ensure re-usable metrics and attributes within the reporting layer. Responsible for creating and maintaining necessary documentation (MDR) to ensure audit readiness where necessary. Prototype improvement ideas. Work effectively with the global team. Expected to play technical leadership as an individual contributor. Articulate challenges, propose and drive solutions ...

\noindent \textbf{Mandatory Skills}\\
Extensive knowledge of Hadoop stack and storage technologies HDFS, MapReduce, Yarn, HIVE, sqoop, Impala, spark, flume, kafka and oozie. Extensive Knowledge on Bigdata Enterprise architecture (Cloudera preferred). Experience in No SQL Technologies (Cassandra, Hbase).

\noindent \textbf{Desired Skills}\\
Experience in Real time streaming (Kafka). Experience with Big Data Analytics \& Business Intelligence and Industry standard tools integrated with Hadoop ecosystem. (R , Python ). Visual Analytics Tools knowledge (Tableau ). Data Integration, Data Security on Hadoop ecosystem. (Kerberos ). Awareness or experience with Data Lake with Cloudera ecosystem.
\end{tcolorbox} 

\captionof{figure}{This is a more detailed version of a sample vacancy posting with only information about the firm and the relevant sub-division redacted.}




\end{document}