\section{Research Setting}\label{sec:research_setting}

\subsection{Context}

This section outlines our research setting within the Information Technology division of a prominent U.S.-based 
financial services institution. The division, comprising multiple specialized units, provides technical support 
ranging from trading systems and wealth management platforms to enterprise-wide initiatives like system 
modernization and compliance reporting. Several distinctive features make this setting ideal for validating 
posting informativeness: precise skill specifications required for technical roles, diverse units generating 
varied requirements, and structured mobility processes yielding both successful and unsuccessful applications.

Within the division, specialized technical roles form the core workforce—application programmers developing 
trading platforms, data analysts supporting risk management, and quality assurance specialists maintaining 
enterprise systems. Each position demands specific technical competencies, from programming languages and 
analytical tools to testing frameworks and development methodologies. Such specialized requirements necessitate 
detailed articulation of skills in job postings, generating rich content for examining whether posting-derived 
measures predict selection outcomes.

A centralized job board facilitates internal mobility, where positions are posted with comprehensive skill 
and responsibility specifications (see Box \ref{box:vacancy_posting} for a representative example). The 
organization's open policy maintains minimal restrictions on internal applications, allowing employees to 
pursue roles aligned with their evolving skills without prior managerial approval. Following application 
submission, recruiting units conduct initial screening and interviews to identify suitable internal candidates. 
External candidates are considered only after a specified timeframe if no appropriate internal candidate emerges.

The organization's emphasis on internal mobility stems from its strategy to retain valuable technical talent 
in a competitive labor market. Enabling transitions into roles matching evolving skills and career aspirations 
helps mitigate the high costs and competitive challenges of external hiring. While HR primarily oversees the 
process in a regulatory capacity, ensuring adherence to organizational policies, leadership increasingly 
recognizes internal mobility's strategic value and seeks data-driven initiatives to optimize these practices. 
This structured approach yields substantial variation in applications and outcomes—essential for validating 
posting informativeness through selection decisions.

\subsection{Data}

Our analysis draws on comprehensive data from the firm's Human Resource Information System (Workday), which 
captures all job requisitions, applications, and selection outcomes. Each record includes complete job posting 
content, application details, and final decisions. A crucial feature for our research design lies in our 
ability to link internal applications to the detailed posting of the applicant's current position at 
application time, enabling direct comparison of skill requirements between current and sought positions.

\begin{table}[t]
   \caption{Job Postings and Internal Mobility (2018-2021)}
   \begin{tabular*}{\textwidth}{@{\extracolsep\fill}lrrr}
   \toprule
   & \multicolumn{3}{c}{Panel A: Yearly Distribution} \\
   \cmidrule{2-4}
   Year & Total Positions & Internal Applications (\%) & Internal Fills (\%) \\
   \midrule
   2018 & 1,326 & 40.3 & 8.3 \\
   2019 & 1,370 & 50.1 & 17.3 \\
   2020 & 1,606 & 75.7 & 42.5 \\
   2021 & 2,240 & 62.5 & 33.9 \\
   \addlinespace
   Total & 6,542 & 58.6 & 27.3 \\
   \midrule
   & \multicolumn{3}{c}{Panel B: Overall Statistics} \\
   \cmidrule{2-4}
   \multicolumn{2}{l}{Total Position Postings (2018-2021)} & \multicolumn{2}{r}{6,542} \\
   \multicolumn{2}{l}{Positions with Internal Applications} & \multicolumn{2}{r}{3,836} \\
   \multicolumn{2}{l}{Positions Filled Internally} & \multicolumn{2}{r}{1,789} \\
   \multicolumn{2}{l}{Pre-2018 Position Postings} & \multicolumn{2}{r}{1,632} \\
   \bottomrule
   \multicolumn{4}{l}{\footnotesize \textit{Notes:} Internal Applications (\%) represents positions receiving at least one internal} \\
   \multicolumn{4}{l}{\footnotesize application. Internal Fills (\%) represents positions filled by internal candidates.} \\
   \label{tab:summary}
   \end{tabular*}
\end{table}

Table \ref{tab:summary} presents our dataset covering individual contributor positions posted between 2018-2021, 
supplemented by 1,632 positions from earlier periods. The observation window captures 6,542 positions, with 
3,836 (58.6\%) receiving internal applications and 1,789 (27.3\%) filled internally. Among positions receiving 
internal applications, we identified 1,370 cases where we could observe the detailed posting of the applicant's 
current position at application time, forming the core sample for our analysis of posting informativeness.

\begin{table}[t]
   \caption{Sample Job Requisitions and Internal Applications}
   \begin{tabular*}{\textwidth}{@{\extracolsep\fill}llllc}
   \toprule
   Job Requisition No. & Current Position & Position Applied To & Date & Selected \\
   \midrule
   REQ2018-103 & App Programmer (Java) & App Programmer (Webservices) & 2018-03 & Yes \\
   REQ2018-245 & Data Analyst & Data Scientist & 2018-06 & No \\
   REQ2019-167 & QA Analyst & Senior QA Analyst & 2019-01 & Yes \\
   REQ2019-389 & App Programmer II (.NET) & Technical Lead & 2019-04 & No \\
   \bottomrule
   \multicolumn{5}{l}{\footnotesize \textit{Notes:} Table represents sample internal applications with current and applied-to positions.} \\
   \multicolumn{5}{l}{\footnotesize Each position contains detailed skill requirements as shown in Box \ref{box:vacancy_posting}.} \\
   \label{tab:requisitions}
   \end{tabular*}
\end{table}

The structure of our data appears in Table \ref{tab:requisitions}, which illustrates how we track internal 
applications across positions. Each application record links to comprehensive job postings detailing required 
skills, experience levels, and responsibilities. Box \ref{box:vacancy_posting} demonstrates the typical depth 
of these postings, showcasing the rich content available for measuring skill requirements.

\begin{tcolorbox}[colback=gray!5,colframe=blue!75!black, 
title=Sample Technical Role Description,
label=box:vacancy_posting]
\small
\textbf{Position:} Application Programmer II (Webservices)

\textbf{Overview:} 
Technical role responsible for design and development of enterprise web services.

\textbf{Key Responsibilities:}
\begin{itemize}
\item Design and implement RESTful web services
\item Develop API documentation and integration guides
\item Optimize service performance and reliability
\item Collaborate with cross-functional teams
\end{itemize}

\textbf{Required Technical Skills:}
\begin{itemize}
\item Java/J2EE development (5+ years)
\item Spring Framework, Spring Boot
\item RESTful Web Services
\item SQL and relational databases
\end{itemize}

\textbf{Additional Requirements:}
Experience with microservices architecture, API security, performance optimization
\end{tcolorbox}

The combination of detailed technical postings, comprehensive observation of application outcomes, and our ability 
to link current and sought positions makes this setting particularly valuable for validating posting informativeness. 
The rich content in job postings reflects precise skill requirements for technical positions. Moreover, access to 
both successful and unsuccessful applications enables robust testing of posting content's predictive power. Our 
unique ability to observe detailed postings for applicants' current positions at the time of application allows 
direct measurement of skill distances, providing compelling evidence of whether these measures meaningfully predict 
selection outcomes.