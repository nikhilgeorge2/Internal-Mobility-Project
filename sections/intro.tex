

\section{Introduction}

Major corporations like Google, Walmart, and IBM, along with other public and private agencies, have committed to 
skills-based hiring policies, focusing on capabilities rather than traditional credentials such as degrees or work 
experience \citep{hbr2022skillsbased, mckinsey2020future, wef2020jobs}. This shift comes at a time when employees 
increasingly pursue less fixed career paths both within and across firms, and when the skills required to perform 
modern work evolve rapidly. These forces compel firms to articulate skill requirements and job responsibilities 
with greater precision in their postings. This information is not only consumed by potential applicants—it is 
embedded in the algorithms used by platforms like Indeed and LinkedIn to curate and match opportunities to 
candidates, and ingested by sophisticated screening tools that shortlist applicant resumes based on fit against 
these requirements. Job postings and their content are also systematically harvested by specialized firms like 
Revelio and Lightcast, who package this data into analytical products that provide insights into workforce skill 
requirements across industries\footnote{The workforce analytics provided by firms like Revelio and Lightcast have 
not only supported businesses but also spurred a growing body of academic research that study trends in skill 
demands and job creation \citep{goldfarb2020artificial, azar2020concentration}, and other critical labor market 
phenomena \citep{hershbein2018recessions, forsythe2020labor, braxton2023technological}.}. These developments 
underscore the growing importance of job postings as a key source of valuable information at the analytical, 
strategic, and policy levels.

Despite this growing centrality of job posting content and significant investment in its analysis, no rigorous validation 
exists of whether posting-derived skill measures predict actual selection and mobility outcomes. While firms are 
investing in more detailed skill descriptions, and sophisticated tools are being built to extract and analyze this content, 
consumers of posting-based analytics and tools operate without clear evidence of informativeness. Legitimate concerns 
exist - postings might be aspirational rather than realistic, could be template-driven or overly generic, and might 
list ``nice to have" skills rather than true requirements. Understanding how informative these postings are for 
measuring skill requirements is crucial for the growing ecosystem of tools and analytics being built on this content. 
The ability to validate posting-derived measures against selection decisions would inform both current applications 
and future innovations.

We partner with the IT division of a major financial services firm to study the informativeness of job postings by 
linking it to actual workers and their mobility process. This setting provides an ideal laboratory - the firm's 
hiring process generates detailed position requirements through job postings, which multiple internal candidates 
consider and apply to. The firm's internal job board offers comprehensive records of both successful and unsuccessful 
applications over multiple years. Crucially, we can link the applicants to the postings of their current job and 
observe both successful and unsuccessful applications, providing the opportunity to assess the informative content 
in job postings to predict selection. By structuring a prediction problem, we can examine whether the skill 
distances derived from posting content meaningfully predict selection outcomes, thereby contributing a first of 
its kind estimate of the nature and level of informativeness in job postings; an information source of increasing 
application inside and outside the firm.

Our analysis reveals substantial predictive power in posting-derived skill measures. The probability of selection 
decreases by 84\% when comparing applications in the lowest versus highest quintile of skill distance, indicating 
posting content captures meaningful differences in worker-job fit. This predictive power is particularly evident 
when examining vacancies with multiple applicants - in 70\% of these cases, the candidate with the shortest skill 
distance was selected. The predictive relationship holds both across vacancies and within applicants (across their 
different applications). Notably, traditional employee characteristics like age, tenure, and work experience 
contribute minimally beyond the skill distance measure - suggesting the rich information in job postings dominates 
standard observable characteristics in explaining selection outcomes.

The first step in our machine learning pipeline converts the key information in a job posting to a job skill-embedding 
by mapping it in the language space of a pre-trained language model. While these embeddings capture rich semantic 
relationships in general language, our context of technical skills and responsibilities likely lies in a much 
lower dimensional space. Moreover, with relatively few selection decisions compared to the embedding dimensions, 
we need dimension reduction to enable effective learning of the distance metric. The distance metric learning, 
akin to fine-tuning the general language model's notion of similarity for our specific context, reveals meaningful 
skill relationships. This learned representation's validity is demonstrated through its strong predictive power on 
held-out selection decisions, and its application reveals novel patterns in how employees navigate opportunities 
based on skill-fit.

We apply the skill distance measure to investigate internal job application patterns, revealing another dimension of 
job posting informativeness. Those whose skills closely match available vacancies adopt a relatively passive approach 
despite higher chances of success, while employees facing larger skill distances actively pursue positions requiring 
skill development. By structuring standard posting and mobility data, we show how posting-derived analytics can generate 
actionable insights about whether reducing search frictions or investing in skill development should be prioritized. 
Such capabilities suggest a promising direction for HR analytics, a field that has seen limited advancement despite 
significant interest \citep{Tambe2019} - the combination of skill information in job postings and mobility patterns 
enables rich workforce analytics. Our analysis reveals novel patterns about internal mobility, providing evidence 
about job search behavior in an environment characterized by employed workers navigating evolving skill requirements 
- a setting distinct from traditional unemployment-focused search models. This ability to measure skill requirements 
from job postings and track mobility decisions opens new possibilities for personnel economics research to 
systematically investigate career navigation and skill development choices.

Our paper illuminates the informative content in job postings by linking their skill and responsibility descriptions 
to selection outcomes and application decisions. Through selection prediction, we provide first validation that 
posting-derived skill measures capture meaningful differences in worker-job fit - a crucial validation given the 
growing use of posting content in platforms, analytics, and research. Through our inquiry into application patterns, 
we illuminate how this posting information relates to mobility behavior. This analysis demonstrates significant 
potential for firms to gain insights about internal mobility and its relationship to skill-fit by leveraging 
posting information, while opening new research possibilities in understanding how employees navigate careers 
in environments characterized by evolving skill requirements.

To summarize, this paper illuminates how findings validate a fundamental data source driving major market decisions 
while demonstrating rich analytical possibilities. By showing posting content contains meaningful skill information 
that predicts outcomes, we inform the growing ecosystem of screening tools and analytics built on this content. The 
results challenge pessimism about HR analytics by showing how posting information can guide specific 
interventions - from reducing search frictions for well-matched workers to targeting skill development where needed. 
This posting-based approach outperforms traditional collaborative filtering methods by leveraging the rich content 
in job descriptions, opening new avenues for research in labor economics and personnel practices.

\textbf{Literature Review}

A rich technical literature has explored approaches to job recommendation, worker-job matching, and skill 
representation learning---supporting platforms that facilitate job search, candidate-opening matching, and 
career trajectory modeling\footnote{The technical literature has included work on recommender systems 
\citep{shaha2012survey,siting2012job} and representation 
learning \citep{heap2014combining, zhu2018person, liu2019tripartite, bian2020learning} to support 
platforms facilitating job search \citep{heap2014combining,giabelli2021skills2job}, matching jobs 
to candidates \citep{zhu2018person,qin2020enhanced} and modeling career paths and skill 
recommendations \citep{maurya2017bayesian, kokkodis2021demand}.}. While these advances demonstrate 
sophisticated ways to process job posting content, they focus on improving platform functionality 
rather than validating whether derived skill measures predict actual selection outcomes. The lack of 
selection data in platform settings provides limited empirical motivation to evaluate the informativeness of 
posting-derived measures.

Two papers studying internal labor markets highlight different approaches to measuring employee-position 
fit. \citet{devos2024data} develops a recommender system for internal positions, using collaborative 
filtering based on observed transitions and employee features to generate position 
recommendations. \citet{2024_Cowgill} studies team assignment mechanisms, measuring fit through 
survey-based skill scores validated by management. While these papers focus on different questions, 
both require some measure of employee fit to new positions. Our approach leverages the rich skill 
information contained in job postings, information these approaches do not utilize, to generate 
validated measures of fit that require neither extensive mobility histories nor expert validation.

Beyond algorithmic applications, a growing literature in management and economics leverages job posting 
text to study labor market dynamics. Researchers have extracted signals from job descriptions to examine 
wage premia \citep{Bana2021}, analyze technology's impact on skill demands \citep{George2024}, and forecast 
effects of generative AI \citep{eloundou2024gpts, 2024_Acemoglu}. Our setting of internal mobility enables 
validation of the informativeness of these posting-derived measures through actual selection decisions. The 
finding that skill-fit strongly predicts selection provides validation for this literature's use of posting 
text to measure skill requirements.

The Information Systems literature has studied algorithms that match workers to jobs and recommend skill 
development \citep{kokkodis2021demand, kokkodis2023good}, while also examining how to assess the value of 
data for algorithmic applications \citep{lei2024value}. These topics connect to our study of how informative 
job postings are about skills and job requirements. Related to our validation focus, 
\citet{raghavan2020mitigating} documents limited transparency about how well automated screening 
systems predict actual selection outcomes - our demonstration that posting-derived skill measures 
strongly predict selection provides a framework for building and evaluating such algorithms.

Our analysis shows how incorporating skill similarity derived from job postings reveals distinct patterns 
in internal mobility. Studying career moves and incentives is a rich literature in management research and 
personnel economics - studies have documented patterns in internal moves \citep{bidwell2024stepping} and 
efficiency gains from internal hiring \citep{bidwell2011paying}, while research spans from early work on 
career incentives \citep{baker1994internal, baker1994wage} to recent studies of internal labor 
markets \citep{tambe2020paying, huitfeldt2023internal}. Job posting content, we demonstrate, is a 
valuable but underutilized source of information that, when combined with mobility data, enables deeper 
understanding of how employees navigate careers in environments requiring regular skill acquisition. The 
patterns in how application intensity varies with skill-fit to emerging positions illustrates the potential 
for enriching our understanding of modern career navigation.

We proceed as follows: In Section \ref{sec:research_setting}, we describe the research setting, the 
firm, data and the internal mobility process. In Section \ref{sec:objective_approach}, we outline 
the conceptual ideas of the prediction problem. Section \ref{sec:machine_learning_pipeline} presents 
the machine learning pipeline, the algorithms and details of data processing and implementation. In 
Section \ref{sec:evaluation_skill_distance}, we evaluate the effectiveness of the skill distance metric 
in predicting selection outcomes in detail. In Section \ref{sec:internal_mobility_patterns} we apply 
the distance measure to new positions to study application intensity to new positions vary with skill 
fit and concluding in Section \ref{sec:conclusion_discussion}.

