\section{Conclusion and Discussion}\label{sec:conclusion_discussion}

Our major contribution lies in providing a rigorous empirical test and validation of the informativeness of job posting content, 
particularly for technical roles in the modern digital economy. Analyzing internal mobility within a major corporation's technology division, 
we demonstrate that skill descriptions in job postings possess significant predictive power for selection decisions. Crucially, our methodology 
leverages information from both the applicant's current role and the sought-after position, revealing that the calculated skill distance strongly 
discriminates between candidates' fit to new opportunities. Indeed, the probability of selection is 84\% higher when comparing applications with 
the closest versus furthest skill distance. This finding, derived from observing both successful and unsuccessful applications, provides robust 
validation of job postings as a meaningful signal of actual skill requirements, at least within the context of technology-related roles in large organizations.

Furthermore, our work illuminates how this validated informativeness of job postings unlocks key insights into internal mobility and 
human capital analytics within the firm. By establishing that posting content reliably reflects skill demands, we demonstrate its value 
as a primary data source for understanding employee transitions and career pathways. The measured skill distances between current and 
potential positions serve as a powerful lens for identifying areas where search frictions impede well-matched candidates and where genuine 
skill gaps exist relative to emerging opportunities. The observed heterogeneity in application intensity, directly linked to skill alignment 
with current vacancies, further underscores how employees actively navigate the internal market, balancing immediate fit with aspirations for 
skill development.

The demonstrated informativeness of job requirements validates and strengthens the broader trend in systematic collection and analysis of job postings data, 
both within and beyond organizational boundaries. This trend, fueled by specialized firms and academic researchers alike, seeks to understand the dynamics of 
the labor market and provide strategic intelligence to firms. Our findings underscore that firms' internal repositories of job postings, alongside increasingly 
available external data on skill requirements across their competitive landscape, represent a valuable, and now empirically validated, resource. This access 
enables a more nuanced understanding of evolving skill demands, informing strategic decisions related to talent acquisition, training, and workforce planning.

Our results suggest firms can move beyond conventional HR metrics to develop validated measures of worker-job fit, leveraging readily available posting data 
to guide specific interventions and foster more effective internal mobility. This capability to link rich posting content to tangible selection outcomes 
firmly positions job postings as a powerful, and often underutilized, source of intelligence for navigating the complexities of human capital management 
in modern, skill-driven labor markets.