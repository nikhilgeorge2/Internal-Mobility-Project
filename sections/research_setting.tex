

\section{Research Setting}\label{sec:research_setting}


\subsection{Context}


This section outlines the organizational setting from which our study draws its data, focusing on the 
Information Technology division of a prominent U.S.-based financial services institution. This division 
provides an ideal setting to validate the informativeness of job posting content through actual selection 
decisions. It is structured into multiple units and subunits designed to meet the comprehensive IT demands 
of a large corporation. Some units specialize in the development and support of IT systems tailored to 
specific business lines, including trading, wealth management, and retail banking. Meanwhile, other 
units focus on overarching IT goals such as system upgrades, legacy system retirement, and compliance 
reporting. This diversity in technical roles generates detailed job postings whose skill requirements 
can be validated against actual selection outcomes.



The diverse and dynamic nature of this IT division, with its wide range of technical and process-driven roles, 
provides a rich context for examining how well job postings capture skill requirements. Teams are continually 
evolving, driven by the need to adapt to emerging technologies and shifting market demands. Internal mobility 
is strategically encouraged to align the evolving competencies of the workforce with the institution's 
operational needs and strategic goals. This backdrop, with its continuous flow of detailed job postings and 
selection decisions, enables us to test whether posting content meaningfully discriminates between 
candidates' skill fit to new positions.


Within these teams, we concentrate on employees classified as individual contributors. These positions demand 
specific technical skills and process knowledge, primarily focused on technical tasks rather than staff management. 
As units continually evolve—whether by developing new systems or enhancing existing ones—there is a frequent need 
for individual contributors who possess emerging, highly specialized skills. This need reflects a broader trend in 
the tech industry, where the demand for specific competencies is rapidly evolving.

Aware of the high external market demand for these technical skills, the organization places significant emphasis on 
promoting internal mobility. This strategy is designed to mitigate the high costs and competitive challenges of 
external hiring. By enabling employees to transition into roles that better match their evolving skills and 
career aspirations, the organization not only retains valuable talent but also fosters a culture of continuous 
learning and adaptation. This internal mobility not only supports individual career development but also aligns 
with the organization’s broader goals of agility and strategic skill alignment.


The internal mobility process is centrally facilitated through an internal job board. When a unit identifies the need 
for a new staff member, the hiring manager crafts a detailed job description specifying required skills and 
responsibilities, which is then posted on this board accessible to all employees. Organizational policy supports 
internal transfers and there are minimal restrictions on applying to internal positions, empowering employees 
to apply for roles that align with their skills and career goals without needing prior approval from their 
current manager. This policy, supports employee mobility and consequently generates a rich set of mobility 
data and selection decisions compared to a firm with more restrictive proactices.

Once applications are submitted, the recruiting unit conducts an initial screening, followed by interviews to 
identify the most suitable internal candidate. If no appropriate internal candidate is identified within a 
specific timeframe, the position is then opened to external applicants. This structured approach underscores 
the organization's commitment to prioritizing and developing internal talent, thereby enhancing retention and 
career progression opportunities for existing staff.

HR's role in this process is primarily regulatory, focused on ensuring adherence to organizational policies 
throughout the recruitment cycle. However, HR leadership is increasingly recognizing the strategic value of 
enhancing internal mobility and is eager to adopt data-driven initiatives that can further this objective. 
They are particularly interested in harnessing the vast amounts of organizational data available to derive 
insights that can refine and optimize internal mobility practices, thus supporting the overarching goal of 
sustained employee development and organizational adaptability.

\subsection{Data}

Our dataset, extracted from the firm's HRIS (Workday), covers the period from 2018 to 2021, capturing 8,180 new job 
positions for individual contributors. For each vacancy, the HRIS tracks shortlisted applicants, distinguishing 
between internal and external candidates. We observe unique Employee IDs for internal applicants and for external 
applicants only if selected. This structure enables longitudinal linking, connecting the vacancy posting of the 
current (internal) applicant to the new vacancy posting. Each job posting includes a text description of the 
division, the required technology skillset, and the associated tasks and responsibilities. 
Table \ref{tab:job_postings} provides a sample of the internal job postings data, illustrating the 
structure of our dataset. Refer to \ref{box:vacancy_posting} for a snippet of the content in a job 
vacancy posting, which demonstrates the typical information included in these listings, such as 
overview, key responsibilities, and required skills.

In our dataset, we observed 1,370 instances of internal applications, resulting in 650 selections and 720 rejections. 
This comprehensive view of internal mobility attempts provides insights into career progression dynamics within 
the organization. Table \ref{tab:job_transitions} presents a sample of internal applications, showing the 
relationship between current jobs and applied positions, along with selection outcomes. The longitudinal 
nature of our data allows us to follow individual employee trajectories, such as tracking an employee selected 
for a position in early 2019 who then applied for different roles in 2021. By linking job descriptions of 
current roles with those of positions applied for, we gain a detailed view of internal mobility patterns over 
time. This linkage is illustrated in both Tables \ref{tab:job_postings} and \ref{tab:job_transitions}, where 
we can observe the progression of employees through different roles and their application outcomes.



\begin{table}[ht]
\centering
\small
\caption{Sample of Internal Job Postings Data}
\label{tab:job_postings}
\begin{tabular}{|c|c|c|c|c|c|c|}
\hline
Req.\_id & Vacancy Posting & Date & Applicant & Internal & Select & Emp\_id \\
\hline
1001 & Data Scientist... & 2018-03-15 & Yes & Yes & No & E5678 \\
1001 & Data Scientist... & 2018-03-15 & Yes & No & No & - \\
1003 & Junior Data Scientist... & 2019-04-10 & Yes & Yes & Yes & E9012 \\
1004 & Software Dev. 1... & 2019-11-01 & No & - & - & - \\
1005 & ML Engineer... & 2020-02-20 & Yes & Yes & No & E3456 \\
\multicolumn{1}{|c|}{...} & \multicolumn{1}{c|}{...} & \multicolumn{1}{c|}{...} & \multicolumn{1}{c|}{...} & \multicolumn{1}{c|}{...} & \multicolumn{1}{c|}{...} & \multicolumn{1}{c|}{...} \\
\multicolumn{1}{|c|}{...} & \multicolumn{1}{c|}{...} & \multicolumn{1}{c|}{...} & \multicolumn{1}{c|}{...} & \multicolumn{1}{c|}{...} & \multicolumn{1}{c|}{...} & \multicolumn{1}{c|}{...} \\
\hline
\end{tabular}
\vspace{0.5cm}
\caption{Sample of Internal Applications}
\label{tab:job_transitions}
\begin{tabular}{|c|c|c|}
\hline
Current Job & Applied Job & Selected \\
\hline
Data Analyst... & Data Scientist... & No \\
Junior Dev... & Software Dev. 1... & No \\
Data Scientist... & ML Engineer... & Yes \\
QA Engineer... & Software Dev. 1... & No \\
ML Engineer... & Senior ML Engineer... & Yes \\
\multicolumn{1}{|c|}{...} & \multicolumn{1}{c|}{...} & \multicolumn{1}{c|}{...} \\
\hline
\end{tabular}
\end{table}



\begin{tcolorbox}[colback=gray!5,colframe=blue!75!black, title=Vacancy Posting Snippet - Hadoop Developer, label=box:vacancy_posting]
\small
\textbf{Overview:} Leading financial institution seeking Hadoop developer for Big Data initiatives.
\textbf{Key Responsibilities:}
\begin{itemize}
\item Develop Big Data strategy and roadmap
\item Design and deploy scalable Big Data platforms
\item Analyze and optimize system performance
\item Evaluate new technologies and innovate solutions
\end{itemize}
\textbf{Required Skills:}
\begin{itemize}
\item Extensive knowledge of Hadoop ecosystem (HDFS, MapReduce, HIVE, Spark, etc.)
\item Experience with Big Data enterprise architecture (Cloudera preferred)
\item Proficiency in NoSQL technologies (Cassandra, HBase)
\end{itemize}
\textbf{Desired Skills:} Real-time streaming, Big Data Analytics, BI tools, Data Lake experience
\end{tcolorbox} 
