\section{Conclusion and Discussion}\label{sec:conclusion_discussion}


We test the informativeness of job posting content by examining how well it predicts selection decisions in the internal 
labor market of a large division at a major firm. Our analysis reveals that skill descriptions in job postings strongly 
discriminate between candidates' fit to new positions - the probability of selection is 84\% higher when comparing 
applications with the closest versus furthest skill distance. While our evidence comes from a single firm, the ability 
to observe both successful and unsuccessful applications provides a unique window into how posting content relates to 
actual selection decisions. This validation complements the growing ecosystem of labor market analytics and research built 
on job posting datasets by demonstrating that posting content captures meaningful differences in worker-job fit.

We transform standard process data - job postings and selection decisions - into actionable insights about workforce 
development. By measuring skill distances between current and potential positions, we identify where search frictions 
impede well-matched candidates and where significant skill gaps exist relative to emerging opportunities. The marked 
heterogeneity in application intensity, linked directly to skill alignment with current vacancies, suggests employees 
actively balance immediate fit against skill development opportunities. This capability to distinguish between matching 
and skill development challenges while revealing how employees navigate evolving skill requirements represents a 
concrete advance in HR analytics.

Beyond validating posting informativeness through selection outcomes, our work demonstrates how firms can develop 
empirically-grounded approaches to workforce analytics. The combination of detailed posting content with observed 
mobility decisions enables systematic investigation of career dynamics within firms. While our findings establish 
the potential of posting-based analytics, realizing this potential requires sustained work in developing frameworks 
that effectively leverage this information source. The growing availability of systematic labor market data suggests 
rich possibilities for combining internal posting measures with external data to deepen our understanding of skill 
requirements and career mobility.

By demonstrating that posting content contains meaningful, predictive information about skills and job requirements, 
we provide empirical support for the growing use of posting-based analytics while illuminating new possibilities for 
research and practice. Our results suggest firms can move beyond conventional HR metrics to develop validated measures 
of worker-job fit that guide specific interventions. This capability to link posting content to outcomes positions 
job postings as a valuable but underutilized source of information for understanding and facilitating internal mobility 
in modern labor markets.

